\newpage

\section{Problem 2 (8 pts)}
\subsection{Description}

Consider the predicate \textit{asks(a, b)} that is interpreted as \textit{``a has asked b out on a date.''}
\begin{enumerate}
\item Translate the following into English: $\forall a \exists b~asks(a, b)$ and $\exists b \forall a ~asks(a, b)$.
\item Can we claim that $\forall a \exists b~asks(a, b) \rightarrow \exists b \forall a ~asks(a, b)$? Discuss in detail.
\item Can we claim that $\exists b \forall a ~asks(a, b) \rightarrow \forall a \exists b~asks(a, b)$? Discuss in detail.
\end{enumerate}

\subsection{Answer}

\begin{enumerate}
  \item $\forall a \exists b~asks(a, b)\rightarrow$ Every person has asked at least one person out on a date.\\
  \\
  $\exists b \forall a~asks(a, b)\rightarrow$ There is one person that have been asked out on a date by everyone else.
  \item  We cannot claim that
  $\forall a \exists b~asks(a, b) \rightarrow \exists b \forall a ~asks(a, b)$. 
  The first predicate states that everyone has asked out at least one person (b) on a date. We have no information
  about the other person. Everyone could have asked someone different or they could have asked the same person. The
  predicate does not provide this information.\\
  The second predicate states that one person was asked out by everyone. Here it is specified that everyone asked out
  the same person.\\
  Therefore,
  $\forall a \exists b~asks(a, b) \rightarrow \exists b \forall a ~asks(a, b)$
   is \textbf{false}.
  \item We can confirm that it is justifiable to claim that
  $\exists b \forall a ~asks(a, b) \rightarrow \forall a \exists b~asks(a, b)$.
  Indeed, claiming that someone have been asked out by everyone ($\exists b \forall a ~asks(a, b)$) implies that 
  everyone has asked out at least one person ($\forall a \exists b~asks(a, b)$).\\
  Therefore, the claim
  $\exists b \forall a ~asks(a, b) \rightarrow \forall a \exists b~asks(a, b)$
  is true.
\end{enumerate}
