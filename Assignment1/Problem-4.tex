\newpage

\section{Problem 4 (12 pts)}
Consider list $\Lambda = \langle w, x, y, z \rangle$, deployed to implement a Queue Abstract Data Type.

\subsection{Enqueue and dequeue operations with head of list as front of queue}

\begin{itemize}
    \item[] enqueue(el, $\Lambda$):  $\Lambda$' = cons($\Lambda$, el)
    \item[] dequeue($\Lambda$): $\Lambda$' = tail($\Lambda$)
\end{itemize}

\subsection{Let head of A correspond to the rear of Queue}

\subsubsection{cons(el, $\Lambda$)}
\begin{enumerate}
    \item[] \textbf{Result:} $\langle el, w, x, y, z \rangle$
    \item[] It's the correct implementation for operation \texttt{enqueue(el $\Lambda$)} since it adds the new element to the rear of the queue (head of $\Lambda$)
\end{enumerate}

\subsubsection{list(el, $\Lambda$)}
\begin{enumerate}
    \item[] \textbf{Result:} $\langle el, \langle w, x, y, z \rangle \rangle$
    \item[] It's \textbf{not} the correct implementation for operation \texttt{enqueue(el $\Lambda$)} since it creates list containing A as a list inside.
\end{enumerate}

\subsubsection{concat(list(el), $\Lambda$)}
\begin{enumerate}
    \item[] \textbf{Result:} $\langle el, w, x, y, z \rangle$
    \item[] It's the correct implementation for operation \texttt{enqueue(el $\Lambda$)} since it adds the new element to the rear of the queue (head of $\Lambda$)
\end{enumerate}
