\newpage

\section{Problem 3 (12 pts)}

\subsection{Description:}

\noindent Let $scientist(x)$ denote the statement ``x is a scientist'', and $honest(x)$ denote the statement ``x is honest.'' Formalize the following sentences and indicate their corresponding formal type.

\begin{enumerate}
\item ``No scientists are honest.''

\item ``All scientists are crooked.''

\item ``All scientists are honest.''

\item ``Some scientists are crooked.''

\item ``Some scientists are honest.''

\item ``No scientist is crooked.''

\item ``Some scientists are not crooked.''

\item ``Some scientists are not honest.''

\end{enumerate}

\noindent Identify pairs that are contradictories, contraries, subcontraries, and pairs that support subalteration (clearly indicating superaltern and subaltern).\\


\subsection{Formalize sentences and indicate formal type}

Let's denote
\begin{enumerate}
    \item[]\textbf{P(x)}: scientist(x): ``x is a scientist''
    \item[]\textbf{Q(x)}: honest(x): ``x is honest''
\end{enumerate}

We can formalize the statements as following

\begin{enumerate}
    \item ``No scientists are honest.'' = $\forall x, (P(x) \to \neg Q(x))$ = E form

    \item ``All scientists are crooked.'' = $\forall x, (P(x) \to \neg Q(x))$ = E form

    \item ``All scientists are honest.'' = $\forall x, (P(x) \to Q(x))$ = A form

    \item ``Some scientists are crooked.'' = $\exists x, (P(x) \wedge \neg Q(x))$ = O form

    \item ``Some scientists are honest.'' = $\exists x, (P(w) \wedge Q(x))$ = I form

    \item ``No scientist is crooked.'' = $\forall x, (P(x) \to Q(x))$ = A form

    \item ``Some scientists are not crooked.'' = $\exists x, (P(x) \wedge Q(x))$ = I form

    \item ``Some scientists are not honest.'' = $\exists x, (P(x) \wedge \neg Q(x))$ = O form
\end{enumerate}


\subsection{Identify pairs that are contradictories, contraries, subcontraries, and pairs that support subalteration}



\subsubsection{Pairs of contradictories}
\begin{itemize}
    \item (3) and (4)
    \item (6) and (4)
    \item (3) and (8)
    \item (6) and (8)
    \item (5) and (1)
    \item (5) and (2)
    \item (7) and (1)
    \item (7) and (2)
\end{itemize}


\subsubsection{Pairs of contraries}
\begin{itemize}
    \item (3) and (1)
    \item (6) and (1)
    \item (3) and (2)
    \item (6) and (2)
\end{itemize}

\subsubsection{Pairs of subcontraries}
\begin{itemize}
    \item (5) and (8)
    \item (7) and (8)
    \item (5) and (4)
    \item (7) and (4)
\end{itemize}

\subsubsection{Pairs that support subalteration}
\begin{itemize}
    \item Subaltern: (5) - Superaltern: (3)
    \item Subaltern: (5) - Superaltern: (6)
    \item Subaltern: (7) - Superaltern: (3)
    \item Subaltern: (7) - Superaltern: (6)
    \item Subaltern: (4) - Superaltern: (1)
    \item Subaltern: (4) - Superaltern: (2)
    \item Subaltern: (8) - Superaltern: (1)
    \item Subaltern: (8) - Superaltern: (2)
\end{itemize}

