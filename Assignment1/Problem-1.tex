\newpage

\section{Problem 1 (8 pts)}
\subsection{Problem:}

\noindent You are shown a set of four cards placed on a table, each of which has a \textbf{number} on one side and a \textbf{symbol} on the other side. The visible faces of the cards show the numbers \textbf{2} and \textbf{7}, and the symbols \textbf{$\Box$}, and \textbf{$\bigcirc$}.\\

\noindent Which card(s) must you turn over in order to test the truth of the proposition that \textit{``If a card has an odd number on one side, then it has the symbol $\Box$ on the other side''}? Explain your reasoning by deciding for \underline{each} card whether it should be turned over and why.\\

\subsection{Answer:}

Let us assume that p represent the statement "A card has an odd number on one side" and the statement q represent "It has the symbol $\Box$ on the other side''. \\ \\
The first card's visible face shows the number 2. The statement reads: \textit{``If a card has an odd number on one side, then it has the symbol $\Box$ on the other side''} (p $\mapsto$ q). Since 2 is not an odd number, whatever symbol is behind that card will not affect the truth of the proposition. Therefore, it does not need to be turned over. In other words, if p is false, then whatever truth value of q (let it be a square of a circle), the proposition of p $\mapsto$ q will always remain true.\\ \\
The second card's visible face shows the number 7. Since this is an odd number, to test the truth value of the stated proposition, this card has to be turned over to check if it has a square symbol on the back. In other words, the statement p is true in this case, therefore the truth value of the proposition p $\mapsto$ q will only be true if the statement q is true. To verify if q is true, the card needs to be turned over to check if it is a square on the back side.\\ \\
The third card's visible face shows the square symbol. In this case, the statement q is true (the card has a square symbol). Therefore, the proposition p $\mapsto$ q will be true regardless of the truth value of the statement p (it can be an odd number or an even number). Therefore, this card does not need to be turned over.\\ \\ 
The fourth card's visible face shows the circle symbol. In this case, the statement q is false (the card has a circle symbol). Therefore, the proposition p $\mapsto$ q will only be true if the truth value of the statement p is true (there is an odd number on the back). Therefore, to verify if p is true, the card needs to be turned over to check if it is an odd number on the back side.\\ \\
In conclusion, the card with visible sides 7 and $\bigcirc$ will be turned over.
