\newpage
\section{Formal Specifications for Active}

\subsection{Mathematical Representation}

\noindent The EFSM of the metro passageway is the tuple $S = (Q, \Sigma_1, \Sigma_2, q_0, V, \Lambda)$, where\\

\noindent $Q = \{Start Setting, Setting, Setting Complete, Washing, Rinse, Spin\}$\\
\noindent $\Sigma_1 = \{cancel, press Button for Cycle Type, set Program, washing Complete, after(3 mins), after(2 mins)\}$\\
\noindent $\Sigma_2 = \{lock Door, unlock Door\}$\\
\noindent $q_0: Setting$\\
\noindent $V: door = \{open, closed\}$\\
\noindent $\Lambda$: Transition specifications\\
\indent 1. $\rightarrow Start Setting$\\
\indent 2. $Start Setting \rightarrow Setting$\\
\indent 3. $Setting \xrightarrow {\text {cancel}} Start Setting$\\
\indent 4. $Setting \xrightarrow {\text {press button for cycle type}} Setting Complete$\\
\indent 5. $Setting Complete \xrightarrow {\text {program set [door is closed] / lock door}} Washing$\\
\indent 6. $Washing \xrightarrow {\text {washing complete}} Rinse$\\
\indent 7. $Rinse \xrightarrow {\text {after(3min)}} Spin$\\
\indent 8. $Spin \xrightarrow {\text {after(2min) / unlock door}} Exit$\\

\noindent The UML state diagram is shown in Figure~\ref{fig:Active}.

\newpage

\subsection{UML state diagrams}

\begin{figure}[h!]
	\centering
		\includegraphics[width=0.8\textwidth]{./figures/eps/metro.eps}
		  \caption{Washing Machine.}
  \label{fig:Active}
\end{figure}

