\newpage
\section{Problem 1: State (7 pts)}

\subsection{Description:}

The declaration of the state of the system is defined by

\begin{itemize}
    \item The set of phone numbers (call it \textit{numbers}) that are recorded in contacts
    \item A record of association between names and phone numbers, given by a correspondence
        (call it \textit{recorded}).
\end{itemize}

\begin{enumerate}
    \item Provide a diagram to visualize the state of the system.
    \item Provide a formal definition for numbers.
    \item Does \textit{recorded} have to be captured by a function? What requirements would a function
enforce? Explain in detail.
    \item What is the domain and the codomain of \textit{recorded}?
    \item What type of function should \textit{recorded} be (full or partial)? Explain in detail.
    \item Will \textit{recorded} be an injective, surjective, or bijective? Explain in detail.
    \item Provide a formal definition for \textit{recorded}.
\end{enumerate}

\subsection{Answer:}

\begin{enumerate}
    \item
    \item
    \item
    \item
    \item As previously stated, the domain of \emph{recorded} is \emph{Numbers}. We know that \emph{Numbers} is the set 
    of all phone numbers that are recorded in contacts, meaning that each element of \emph{Numbers} has to be recorded in 
    contacts and therefore has a name associated to it. We can conclude that \emph{recorded} is defined for all elements 
    of its domain, namely \emph{Numbers}. Thus, \emph{recorded} is a full or total function.
    \item We know that each element of \emph{Name} (the codomain) has to be associated with at least one element of 
    \emph{Numbers} (the domain). We also know that a name can be associated with multiple phone numbers. This type of 
    function is described as a surjective function. 
    \item The function \emph{recorded} can be formally defined as:\\
    $\{ \forall y \in \emph{Name}  \exists x \in \emph{Numbers} | recorded(x) = y \}$
\end{enumerate}
